
\subsection{Problem definition}

High global ambitions for decreased CO2 emission and the resulting increase in implementation of renewable energy sources, introduce higher demands on the energy grid than ever. The volatile nature of renewable energy sources, drives the need for collaboration between countries, energy sectors, and energy sources, to handle peak loads and hours of energy scarcity. Therefore, the need for analysis tools providing insights in the constraints and possibilities decision makers must deal with, has never been more present.  \\

A frequently used tool to gain insight in the future energy grid compositions, is energy-economic models on either regional, national or international scale. These models do however suffer from large uncertainties and a lack of validation possibilities, resulting in unreliable and therefore less informative results. 

Model uncertainty can be categorized as either parametric uncertainty, arising from uncertainty in input parameters and data, or as structural uncertainty introduced by an incomplete or faulty mathematical description of the problem at hand \cite{DeCarolis_MGA}. Structural uncertainty is however not caused by the modelers lack of mathematical talent, but is the result of a dealing with a very complex problem, influenced by multiple actors such as policymakers and private company's in the energy sector. 



Developing the future energy supply of Europe is a process highly influenced by political decision makers, and as the theoretical optimal power supply for Europe, might not be complying with politics, it is valuable to explore alternative near-optimal solutions. \\

- Increased demand and a need to reduce CO2 emmisions 
- Fundamental change is needed (policy wise)
- Energy-economy models is an important tool 
- Modelers should focus on robust insights rather than point estimates
- Uncertainty in the models (Structural and parametric)
- Structural uncertainty is addressed with higher comlexity
- Parametric uncertainty is addressed with running multiple scenarios or sensitivity analysis
- Scenario approach does not include less expected real-world developments 
- Little to none possibility to validate models 



\subsection{Project description}

This project will explore the use of the so-called Modeling to Generate Alternatives(MGA) \cite{MGA} formulation, to find a range of alternative near cost optimal configurations of the European power supply. The working model of the European energy grid build in PyPsa \cite{Pypsa}, presented in: \cite{PypsaModel}, will serve as the foundation of this project. The MGA approach will build on top of this model, however, only including major technologies available in the energy sector, such as solar, wind, and fossil fuel power plants.

- MGA \cite{Brill_MGA_1982}
- Feasible near optimal decision space 


Alternative solutions generated with energy-economy
optimization models also provide valuable insight that can be used to
challenge preconceptions and suggest creative alternatives to decision makers.

\subsection{project boundary's}


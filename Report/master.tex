\documentclass[a4paper, 12pt]{article}
\usepackage[utf8]{inputenc}
\usepackage{amsmath}
\usepackage{amsfonts}
\usepackage{amssymb}
\usepackage{graphicx}
%\usepackage{wrapfig}
\usepackage{csquotes}
\usepackage[left=25mm,top=25mm,right=25mm,bottom=25mm]{geometry}
%\usepackage{floatrow}
\setlength{\parindent}{0em}
\usepackage[font=footnotesize,labelfont=bf]{caption}
\numberwithin{figure}{section}  
\numberwithin{table}{section}
\usepackage{enumitem}
\usepackage{subcaption}
\usepackage{float}
\usepackage{url}
\usepackage{fancyhdr}
\usepackage{array}
%\usepackage[nottoc,numbib]{tocbibind}
\usepackage[pdfpagelabels=true]{hyperref}
\usepackage[font=footnotesize,labelfont=bf]{caption}
\usepackage[T1]{fontenc}
\usepackage {palatino}
\usepackage{listings} %für Quellcode
%\usepackage[numbers,super]{natbib}
%\usepackage{textcomp}
\usepackage{chemmacros}
\usepackage[version=4]{mhchem}
\usepackage{longtable} % wegen diesen langen tabellen
\usepackage{array,multirow} %für tabelle
\usepackage{graphicx}
\usepackage[nottoc,numbib]{tocbibind}
\usepackage{verbatimbox} %für die IGOR Skripte
\usepackage{verbatim}
\usepackage{epsfig}
\usepackage{tabularx}
\usepackage{booktabs}
\newcolumntype{Y}{>{\centering\arraybackslash}X}
\usepackage{soul}
\usepackage{xcolor}
\usepackage{titlepic}
\newcommand{\mathcolorbox}[2]{\colorbox{#1}{$\displaystyle #2$}}
\usepackage[backend=bibtex, style=chem-angew, backref=none, backrefstyle=all+]{biblatex}
\bibliography{references.bib}
\defbibheading{head}{\section{Bibliography}\label{sec:Lit}} 
%ersetzt \section{Literatur} und sieht schön aus

\let\cite=\supercite % Zitate hochgestellt


\usepackage[utf8]{inputenc} %to post code
 
\usepackage{listings}
\usepackage{color}
 
\definecolor{codegreen}{rgb}{0,0.6,0}
\definecolor{codegray}{rgb}{0.5,0.5,0.5}
\definecolor{codepurple}{rgb}{0.58,0,0.82}
\definecolor{backcolour}{rgb}{0.95,0.95,0.92}
 
\lstdefinestyle{mystyle}{
    backgroundcolor=\color{backcolour},   
    commentstyle=\color{codegreen},
    keywordstyle=\color{purple},
    numberstyle=\tiny\color{codegray},
    stringstyle=\color{codepurple},
    basicstyle=\footnotesize,
    breakatwhitespace=false,         
    breaklines=true,                 
    captionpos=b,                    
    keepspaces=true,                 
    numbers=left,                    
    numbersep=5pt,                  
    showspaces=false,                
    showstringspaces=false,
    showtabs=false,                  
    tabsize=2
}
 
\lstset{style=mystyle}

\title{\Huge An exploration of MGA methods for use in strategic energy planning}
\author{ Tim T. Pedersen}
\date{ January 2020}
\titlepic{\includegraphics[width=.7\linewidth]{Images/placeholder.png}}

\begin{document}

\maketitle

\clearpage

\tableofcontents
\clearpage

\chapter{Introduction}


%Starting your introduction
% Large Scale Background
As climate change and the effects thereof has become more and more evident in recent years, the effort to reduce the emission of greenhouse gases has increased. This come to show in ambitious goals of reduced emissions on national and international scale. The majority of greenhouse gas emissions arise from the energy sector including transportation, heating/cooling and electricity demand. 

%Topic and context

In order to achieve reductions in greenhouse gas emission from the energy sector, large amounts of variable renewable energy sources, such as wind and solar energy is being implemented in the energy grid. The volatile nature of these variable renewable energy sources, drives the need for collaboration between countries and energy sectors to handle periods of scarce resources and fluctuations in energy demand. 
This complicates the already complex task of energy system synthesis even further, hereby requiring decision makers to have greater in depth knowledge. Therefore, the need for analysis tools providing insights in the constraints and possibilities of the complex energy systems, has never been more present.  

% Energy economic models 
A frequently used tool to gain insight in future energy grid compositions, is numeric techno-economic energy models on either regional, national or international scale. These models can be used to study the behavior and composition of existing and future energy networks, together with the impact of new technologies or structural changes in the networks \cite{Gorm_impact_of_CO2_PYPSA}. The models often approximates the behavior of the physical systems, through a series of mathematical constraints based on energy laws. By inputting time series's for wind and solar availability and energy demand, the behavior of the energy network can be studied with alternative configurations of energy generating and storage technologies. Using an optimization algorithm, it is possible to find the cheapest configuration of the energy network that satisfies all user prescribed constraints. 

% Model uncertainties 
Techno-economic energy models do however, suffer from large uncertainties and the lack of validation possibilities, resulting in unreliable and therefore less informative results.
Model uncertainty can either arising from uncertainty in input parameters and data, such as uncertainties in future technology prices and energy demand. This type of uncertainty is referred to as parametric uncertainty. Parametric uncertainty is well understood and often studied with sensitivity analysis or Monte Carlo simulations. 
A different type of uncertainty introduced by an incomplete or faulty mathematical description of the problem, has been found to have just as large effects on the results produced by these techno-economic models \cite{DeCarolis_MGA}. As this type of uncertainty relates to the mathematical foundation or structure of the model, they are referred to as structural uncertainty. 
Structural uncertainty is inevitable as it is impossible to create completely accurate mathematical models representing physical systems. Furthermore, the economy of energy systems, which is often the parameter to be optimized, is highly influenced by politics and public opinion. It is easy to imagine that a scenario with large amounts of aerial electricity transmission lines would be met with public resistance, and therefore a scenario with less transmission at the same cost might be favorable. 
This does in term mean that the model objective is not only to reduce system cost, but also to satisfy as many of the involved parties as possible. It is however impossible to model the satisfaction of involved parties, let alone the challenge of identifying all future parties involved.  

%Focus and scope

% Literature review/Current standpoint
Recently an approach addressing the structural uncertainty of the techno-economic models has been proposes by J. DeCarolis. In an article \cite{DeCarolis_MGA} a technique called Modeling to Generate Alternatives (MGA), from the field of management research/planning science \cite{Brill_MGA_1982}, is applied to the field of energy planning. The root cause of structural uncertainty cannot be addressed as it can with parametric uncertainty, as the origin of structural uncertainty is hard to define. Instead one must investigate all solutions near the one found to be optimal, and estimate the likelihood of these near optimal solutions being the true optimal solution. In the technique proposed by J. DeCarolis, a finite set of maximally different near-optimal solutions are found. The difference in the found solutions can then be used as a measure of structural uncertainty and provides a variety of alternatives to the optimal configuration of the energy system. The concept of using MGA algorithms on energy planning problems have been further studied and the result presented in a range of articles and papers; \cite{DECAROLIS2016}, \cite{MGA_Price}, \cite{BERNTSEN2017886}, \cite{Yavuz2011}, \cite{Optimum_not_enough} and \cite{Fabian_MGA}.

The MGA technique introduced by \cite{Brill_MGA_1982} and implemented on an techno-economic energy model by \cite{DeCarolis_MGA}, is referred to as the Hop Skip Jump (HSJ) MGA algorithm. It will produce a small number of alternative near optimal solutions, by altering the objective function of the optimization algorithm. Instead of minimizing cost, the objective is changed such that the model seeks to implement previously unused technologies. In order to ensure that the found alternative solutions are near the optimum, a new constraint is added to the model. This constraint allows for the new MGA solutions to deviate in cost by a certain amount from the optimal solution. This constraint is referred to as the MGA constraint. 

The alternative solutions found with the HSJ MGA algorithm, do provide some insights in the characteristics of all near optimal solutions, but it is not possible to determine if all possible solutions have been found. 
Furthermore, the HSJ method of finding alternative  solutions is somewhat random and the found alternative solutions are highly dependent on the starting point. In order to fully understand the characteristics of all near optimal solutions, a more structured method is needed.  


% Model concept
In this project the model presented in: \cite{PyPSA_euro_30_model} of the European electricity grid, will serve as a base model, to use for testing and validation. The model is build in the open source framework PyPSA\cite{Pypsa}, and formulated as a techno-economic linear optimization problem. The objective in the model is to minimize total annual system cost, while satisfying a range of constraints ensuring feasible operation. The model groups the European electricity network into 30 nodes, each one representing a single country. Countries are linked with power lines approximating the current layout of the European transmission grid. Each node in the network, will in this project, only be granted access to three electricity generating technologies and no storage technologies, simplifying the network drastically compared to the configuration used in \cite{PyPSA_euro_30_model}. The energy generating technologies chosen are open cycle gas turbines (OCGT), wind and solar photovoltaic (solar PV).


%Relevance and importance

The current use of techno-economic models combined with optimization tools, not using MGA algorithms, provide a very rigid solution to the future configuration of energy networks. Usually a single optimal solution is found, and used as an end goal to strive towards. This method of doing energy planning provides little guidance, in the case of unforeseen events and changing objectives. Using the MGA approach presented by J. DeCarolis, it is possible to have a set af alternative near optimal solutions providing guidance if the initial optimal solution, should become unachievable or unfeasible. If the characteristics of all near optimal solutions was made available, it would be possible to define a tolerance on the optimum, and to provide details on minimum required capacities and must have technologies. Furthermore, it would be possible to provide a cost estimate for deviating from the optimal design, guiding decision makers in the case of unforeseen events and changing objectives. Having this information would add value to techno-economic optimization studies, as the result would be relevant for longer periods of time, requiring less frequent reevaluations of the studies. 


%Questions and objectives
In this project the MGA concept will be further explored, in an attempt to map the entire volume of near optimal solutions, providing a detailed description of all feasible near optimal configurations of a given techno-economic model. The objective of this project is to develop a structured method of investigating the characteristics of all feasible near optimal solutions to a an techno-economic optimization problem. The developed method should be verified on a techno-economic model and the quality of the insights provided should be discussed. 



%Overview of the structure
Initially a detailed analysis of the mathematical formulation of a techno-economic model will be presented. The properties of the mathematical constraints used in the model is analyzed together with the objective function. Having established the formulation of the model, the characteristics of the set containing all near optimal solutions will be presented, and the use of optimization algorithms used to find optimal solutions in this set will briefly be discussed. With a good understanding of the techno-economic optimization problem, the working principles of existing MGA algorithms will be introduced followed by a presentation of the MGA method developed in this project. 
In chapter 3, the techno-economic model used in this project will be presented, explaining the implemented technologies and accounting for input data. The results of the computational experiments performed in this project, using the reference model and the presented MGA algorithm, will be presented in chapter 4. Several experiments have been performed highlighting the flexibility and usability of the proposed MGA algorithm. The found results will discussed in chapter 5, analyzing the characteristics of the techno-economic model of Europe used in this project, together with an evaluation of the MGA algorithm itself. Finally the project will be concluded upon, highlighting relevant findings and achieved objectives. 




\clearpage

\chapter{Methodology}


\section{Model}
	
	- Network layout 
	
\begin{figure}[H]\centering
	\includegraphics[width=0.95\textwidth]{./Images/network_layout}
	\caption{Network layout}
	\label{fig:network_lay}
\end{figure}
	
\section{The optimization problem}\label{sec:OptimizationProblem}


The optimization problem at hand is a simplified energy economic model of Europe, build with focus on exploring the composition of VRES (variable renewable energy sources) on a global and national scale. In the model each country is represented as a note connected to the surrounding countries through a link. Each country has three energy producing technologies available, gas, wind and solar power. A data resolution of 1 hour is used, and simulations run over an entire year. 

Following the naming convention from \cite{PyPSA_euro_30_model}, indexing the notes in the network with the variable $n$, the power generating technologies by $s$, the hours in the year by $t$ and the possible connecting power lines by $l$, the contributing variables to the objective function describing the total annualized system cost is the following: 

\begin{itemize}
	\item Hourly dispatch of energy from the given plants in the given countries $g_{n,s,t}$ with the marginal cost $o_{n,s}$.
	\item Total installed capacity of the given technologies in the given countries $G_{n,s}$ with the capital cost $c_{n,s}$.
	\item Total installed transmission capacity for all lines $F_{l}$ with the fixed annualized cost $c_{l}$.
	
\end{itemize}

The objective function for the optimization problem then becomes: 

\begin{equation}
min \left( \sum_{n,s} c_{n,s} G_{n,s} + \sum_l c_l F_l + \sum_{n,s,t} o_{n,s} g_{n,s,t} \right)
\end{equation}{}

This objective function is subject to a range of constraints ensuring realistic behavior of the system. As described in \cite{PyPSA_euro_30_model} a power balance constraint is issued to ensure stable operation of the network. These constraints force the sum of energy produced and consumed in every hour to equal zero. The hourly electricity demand at each node is described by $d_{n,t}$, the incidence matrix describing the line connections is given by $K_{n,l}$ and the hourly transmission in each line is described as $f_{l,t}$. Then the power balance constraint becomes:

\begin{equation}
\sum_s g_{n,s,t} - d_{n,t} = \sum_l K_{n,l} f_{l,t} \; \forall n,t
\end{equation}

For all conventional generators the maximum hourly dispatch of energy is limited by the installed capacity. It is important to node that for all simulations performed in this project the installed capacity is a variable. 

\begin{equation}
0\leq g_{n,s,t} \leq G_{n,s} \; \forall n,s,t
\end{equation}

The dispatch of variable renewable energy sources (wind and solar) is not only limited by the installed capacity, as availability, hence the name, is variable. Therefore the constraint for dispatch of variable renewable energy sources become:

\begin{equation}
0 \leq g_{n,s,t} \leq \overline{g}_{n,s,t} G_{n,s} \; \forall n,s,t
\end{equation}

Where $\overline{g}_{n,s,t}$ represents the normalized availability per unit capacity. 

The installed capacity is constrained by the geographical potential calculated in \cite{PypsaModel}.

\begin{equation}
0 \leq G_{n,s} \leq G_{n,s}^{max} \; \forall n,s
\end{equation}

All transmission lines in the model modelled with a controllable dispatch constrained by the fact that there must be energy conservation at each node the line is connected to. !! Something here about which lines is included !!!! . Furthermore the transmission in each line is limited by the installed transmission capacity in each line. 

\begin{equation}
|f_{l,t}| \leq F_l \; \forall l,t
\end{equation}

In the model it is possible to activate a CO2 constraint, limiting the allowed CO2 emissions for the entire energy network. As in \cite{PypsaModel} the constraint is implemented using the specific emissions $e_s$ in CO2-tonne-per-MWh of the fuel for each generator type $s$, with the efficiency $\eta_s$ and the CO2 limit $CAP_{CO_2}$. 

\begin{equation}
\sum_{n,s,t} \frac{1}{\eta_s}g_{n,s,t} e_s \leq CAP_{CO_2}
\end{equation}

The model is implemented in the open source software PyPSA \cite{Pypsa}, using much of the software presented in \cite{PypsaModel}. Optimization of the model is performed with the optimization software Gurobi \cite{Gurobi}. 

\section{Properties of the near optimal feasible space}\label{sec:properties_of_hull}

Analyzing the original optimization problem one can deduct that the feasible decision space, must be convex, as all constraints $f_i$ and the objective function $f_0$ satisfy equation \vref{eq:convex_requirement}, and therefore must be convex \cite{ConvexOpimization}. 

\begin{equation}\label{eq:convex_requirement}
f_i(\alpha x + \beta y) \leq \alpha f_i(x) + \beta f_i(y) \; \forall \; x, y \in \mathbb{R}^n and  \; \alpha, \beta \in \mathbb{R}
\end{equation}

Furthermore, when all variables are bounded; hourly production by the power balance constraint and installed capacity by geographical potential, the feasible decision space is not only convex but also closed. If the geographical potential constraint is excluded the feasible decision space becomes an open convex space as illustrated on \vref{fig:sketch_feasable_space}, this does however not have any immediate consequences, as the objective function increases as one moves in the open direction of the space. 

\begin{figure}[ht]
	\centering
	\incfig{Feasible-space}
	\caption{A sketch of a one dimensional feasible space with MGA constraint }
	\label{fig:sketch_feasable_space}
\end{figure}

\begin{equation}
W = \{ \vec{x}\in \mathbb{R}^d | f_i(x) \geq 0 \}
\end{equation}
!!! This is not completly right !!! 


It is important to note that the variables $\vec{x}$ that defines the decision space in the original solutions are, all hourly technology dispatches $g$, all installed capacities $G$ and all installed line capacities $F$. 

\begin{equation}
\vec{x} = \{g_{n,s,t} \wedge G_{n,s} \wedge F_l \; \forall \; n,s,t,l \}
\end{equation}

Therefore, the dimensionality, of the decision space must be given by the number of nodes in network $n$ for every technology $s$ for every hour $t$, plus the number of nodes $n$ times technologies $s$ and finally the number of lines $l$  \vref{eq:dimentionality}.

\begin{equation}\label{eq:dimentionality}
d = n\cdot s \cdot t + n\cdot s + l
\end{equation}


In the case of the reference model used in this project that gives 
$ 30 \cdot 3 \cdot 8765 + 30 \cdot 3 + 90 = 789030$ 
!!! number of lines is a guestimate!!

The true dimensionality might be lower, as some variables do have strong corelations. 

\subsection{Sub space}

As the dimensionality of the decision space is very large, and therefore becomes very unhandy to work with, it makes sense to look at a subspace of lower dimensionality. One could choose to ignore the hourly dispatch of energy from the individual generators, hereby reducing the dimensionaly by a substantial amount. 

\begin{equation}
d^* = n\cdot s + l
\end{equation}

In that case the dimensionality would only be $d^* = 30\cdot 3 + 90 = 180$. 

The subspace would then be given by:

\begin{equation}
W^* = \{\vec{x}^* \in \mathbb{R}^{d^*} |    \}
\end{equation}

The set $W^*$ therefore includes information about installed capacities of all technologies and transmission lines. Since plant operation, is not the focus of this project, but rather distribution of capacities,  the subspace $W^*$ still provides the information of interest, despite its much lower dimensionality. 

\subsubsection{Further reduction}
If desired it is possible to further reduce dimensionality, by sacrificing all spatial information.

\begin{equation}
x^{**} = \{\sum_n G_{n,s} \forall n,s  \}
\end{equation} 

\begin{equation}
d^{**} = s
\end{equation}

\begin{equation}
W^{**} = \{\vec{x}^{**} \in \mathbb{R}^{d^{**}} |    \}
\end{equation}



\section{Modeling to Generate Alternatives (MGA)}\label{sec:MGA}
In this section the basic principles of MGA will be explained together with the benefits and challenges this technique introduces. 

\subsection{Motivation for using MGA}

In the field of mathematical modeling, the scientist aim to produce models representing physical systems as realistically as possible. However, some degree of uncertainty in the models is inevitable as model fidelity is limited by a range of factors including: numeric precision, uncertainty of data, model resolution etc. Modeling of energy systems is a field especially prone to large model uncertainties, deriving not only from lack of fidelity, but from factors such as unmodeled objectives and structural uncertainty \cite{DeCarolis_MGA}. 

The MGA approach was first introduced in 1982 by Brill et al. \cite{Brill_MGA_1982}, in the field of operations research/management science. This is a field where unmodeled objectives and structural uncertainty, are highly influential. 

!! CITATION !!
The basic insight can be
summarized as follows: Because it is not possible to develop a complete
mathematical representation of complex public planning problems,
structural uncertainty in optimization models will always exist. As a
result, the ideal solution is more likely to be located within the model's
inferior region rather than at a single optimal point or along the noninferior frontier (Brill, 1979)

Policy makers often have strong concerns outside the scope of most models
(e.g., political feasibility, permitting and regulation, and timing of
action), which implies that feasible, suboptimal solutions may be
preferable for reasons that are difficult to quantify in energy economy
optimization models.

The purpose of MGA is to efficiently search the feasible
region surrounding the optimal solution to generate alternative
solutions that are maximally different. !!!

\subsection{Technical explanation of MGA HSJ}

The MGA technique was first introduced in 1982 by Brill et. al in the article \cite{Brill_MGA_1982} and later rediscovered by DeCarolis in \cite{DeCarolis_MGA} for use in energy system optimization. The tecnique lets the user search the near optimal feasible decision space for an optimization problem such as the one addressed in this project described in \ref{sec:OptimizationProblem}. 

In section \ref{sec:OptimizationProblem} a series of constraints bounding the network model is listed. Together these constraints form a feasible region that can be described as a convex set in a $d$ dimensional space. Where d is the number of variables in the model. The feasible set is convex as all bounding constraints are linear. The fact that linear constraints form a convex set is shown in \cite{ConvexOpimization}. The MGA technique introduces yet another constraint limiting the size of this convex set even further by limiting the objective function value of all feasible points to be within a certain range of the optimal solution. The goal of the MGA technique is to explore a finite set of alternative solutions located within this convex set. 

In the orginal articel by Brill et. al \cite{Brill_MGA_1982} the HSJ MGA technique is descrbed with the following steps. 

(1) obtain an initial optimal solution for the problem at hand; (2) define a target value for the objective function by adding a user specified amount of slack to the value of the objective function in the initial solution (3) introduce the constraint limiting the objective function to surpass this target value, to the model (4) formulate a new objective function that seeks to minimize the sum of decision variables that had non zero values in the previous solution of the problem (5) iterate the reformulated problem, updating the objective function every time (6) terminate the optimization when the new solution is similar to or close to any previously found solution. Step 3 and 4 was described mathematically in \cite{Brill_MGA_1982} as follows:

\begin{equation}
\begin{split}
Minimize :&  p = \sum_{k \in K} x_k \\
Subject to :&  f_j(\vec{x}) \leq T_j \forall j  \vec{x}\in X
\end{split}
\end{equation}

In this formulation $k$ represents the variable indices for the variables with nonzero values in the previous solution, $j$ is the objective function indices if multiple objective functions exists, $f_j(\vec{x})$ is the evaluation of the $j$'th objective function and $T_j$ is the target value specified for the particular objective function. In the formulation of the constraint $\vec{x}\in X$ specifies that all previously defined constraints still applies as all new solutions $\vec{x}$ must be a part of the set of feasible solution vectors from the original formulation $X$.

How the new objective function precisely is formulated and which variables to include is discussed in \cite{DECAROLIS2016}, where two alternative approaches of defining the new objective function is presented. One approach suggest giving all nonzero variables from the last iteration a weight of 1 in the new objective function. This approach does not consider weight from previous iterations. However, the second approach suggests adding on to the coefficient with a factor of +1 for every time one variable has appeared with nonzero in a row, hereby further increasing the intended to reduce the use of that specific technology. This 



\subsection{Other MGA approaches}

\section{Novel MGA approach}

In this section a novel approach towards MGA optimization of energy networks will be presented. Based on the same concepts as presented in \ref{sec:MGA} this method seeks to explore not only a few alternative solutions from the decision space, but the entire decision space. Hereby an in depth knowledge of the possible solution is obtained providing insight in the distribution of alternative solutions.

An important feature about the method developed is that it can be used for any dimensional decision space. 

The method developed can be divided into two phases. In the first phase, the shape of the feasible near optimal decision space is found, and in the second phase relevant data is extracted from the found space. 

\subsection{Decision space mapping}
As explained in section \vref{sec:properties_of_hull}, the near optimal feasible space will always be convex, and can either be closed or not. However, when the MGA constraint from equation \vref{eq:MGA_constraint} is introduced the space will be closed. 

\begin{equation}\label{eq:MGA_constraint}
f(\vec{x}) \leqslant f(\vec{x}^*) \cdot (1+\epsilon)
\end{equation}

As we now have a closed convex space, it now is possible to explore the shape of this convex set. Assuming that all constraints used including the MGA constraint is linear, the convex set must be a polyhedral and therefore it is possible to define the shape of this set with a finite number of vertexes. !!! This might not be the case for CO2 constraint!!!! \\

However, finding these vertices is no trivial task. In the method developed, all solutions found, that lie within the near optimal feasible space is treated as a point in that space. Furthermore, the possibility of letting the objective function search in a given direction in the decision space is utilized, by replacing the original objective function to an objective function on the from presented in \vref{eq:objective_func_face_normal}.

\begin{equation}\label{eq:objective_func_face_normal}
Minimize \; p = \vec{n}_i\vec{x}
\end{equation}

Where $\vec{n}_i$ is the $i$'th normal vector. 


The method proposed here will use the following steps to approximately find all vertices. 

\begin{enumerate}
	\item Find initial solution
	\item Add MGA constraint
	\item Maximize and minimize all variables
	\item Based on these points define a convex hull, and define all face normals
	\item Iterate over each face normal and change objective function to \vref{eq:objective_func_face_normal}
	\item Add the newly found points to list of points and define new hull and its face normals 
	\item Repeat step 5 and 6 until the size of the convex hull converges 
\end{enumerate}

The convergence criteria used in this project is that the hull size must not increase by more than 2\% in two consecutive iterations. 

The result of following these steps is a list of points defining a hull in $d$ dimensional space, however this on its own does not provide much usefull information. To gain any knowledge about the network being analysed on must follow the steps provided in part two of this method. 

\subsection{Hull fill}










\subsection{Pseudo code}

\begin{itemize}[label={}]
	\item Solve network subject to regular constraints and with original objective function
	\item Add MGA constraint !Equation number
	\item while $\epsilon>tol$
	\begin{itemize}[label={}]
		\item If first loop
		\begin{itemize}[label={}]
			\item directions = max and min all variables
		\end{itemize}
		\item Else
		\begin{itemize}[label={}]
			\item directions = normals to hull faces
		\end{itemize}
		\item for direction in directions
		\begin{itemize}[label={}]
			\item objective function = direction[i] * variable[i]
			\item point on convex hull += solve problem subject to objective function
		\end{itemize}
		\item hull = ConvexHull ( points on convex hull)
		\item $epsilon$ = new hull volume - old hull volume / hull volume
	\end{itemize}
	\item Evenly distribute points in hull 
	\item Plot histogram using evenly distributed points. 
\end{itemize}

\section{Implementation and utilization of parallel programming}







\clearpage



\section{Notes}

\subsection{TO DO}

\begin{itemize}
    \item MGA theory
\end{itemize}{}

\subsection{Python Packages used}

\begin{itemize}
    \item import\_ipynb
    \item \$ pip install import\_ipynb
    \item This package is used for importing other ipython (jupyter) notebooks in to a second notebook
    \item ------
    \item 
\end{itemize}{}

\clearpage



\section{Notes on references}

\subsubsection{Impact of CO2 prices on the design of a highly decarbonized coupled
electricity and heating system in Europe\cite{PypsaModel}}
An investigation on the CO2 price levels needed to reduce CO2 emissions. In the article a PyPSA model of Europe is presented. The model could be used in this project. 

\subsubsection{MODELING TO GENERATE ALTERNATIVES: THE HSJ
APPROACH AND AN ILLUSTRATION USING A
PROBLEM IN LAND USE PLANNING \cite{Brill_MGA_1982}}

This is the original article, \cite{Brill_MGA_1982}, explaining the thoughts behind MGA. In this article the HSJ (Hop Skip Jump) approach is implemented. This article seams to be the mother of all other MGA articles. 

\subsubsection{MGA: a decision support system for complex, incompletely defined problems\cite{Brill_MGA_1990}}
Elaborating on the MGA approach presented in \cite{Brill_MGA_1982}, and evaluating the performance of MGA as a whole. 

\subsubsection{Using modeling to generate alternatives (MGA) to expand our thinking on
energy futures\cite{DeCarolis_MGA}}

\cite{DeCarolis_MGA} is one of the first implementations of MGA on energy planning. Uses the HSJ method from \cite{Brill_MGA_1982}.

\subsubsection{Modeling to generate alternatives: A technique to explore uncertainty
in energy-environment-economy models \cite{MGA}}

In this article MGA is used to explore near optimal solutions in energy network optimization, much like \cite{DeCarolis_MGA}. However a slightly more advanced MGA objective function is used. The objective function to be maximized is the Manhattan distance between the current and all preveiously generated MGA solutions. 

\subsubsection{Ensuring diversity of national energy scenarios: Bottom-up energy
system model with Modeling to Generate Alternatives \cite{BERNTSEN2017886}}
A different approach towards implementing MGA on energy system planning. Here they use the  EXPANSE software/model to implement MGA on. They use a sort of random search MGA approach.

\subsubsection{Simulation-Optimization techniques for
modelling to generate alternatives in waste
management planning\cite{Yavuz2011}}
This article describes the MGA method used in \cite{BERNTSEN2017886}. Here a random population is created and is sorted through a number of itterations. 

\subsubsection{GENETIC ALGORITHM APPROACHES FOR ADDRESSING UNMODELED OBJECTIVES IN OPTIMIZATION PROBLEMS \cite{Genetic_Algorithms_for_MGA}}
This article describes the basic theory of MGA very well, and introduces two new genetic algorithms, that could be used for MGA. The Algorithms are based on genetic nieching/sharing algorithms. 

\subsubsection{A Co-evolutionary, Nature-Inspired Algorithm for the Concurrent Generation of Alternatives \cite{FireFly_MGA_Article}}

The article \cite{FireFly_MGA_Article} describes an implementation of the genetic firefly algorithm used to perform MGA. 

\subsubsection{Swarm Intelligence and Bio-Inspired Computation : Theory and Applications - Chapter 14 \cite{Bio_computation_book}}

The book \cite{Bio_computation_book} Chapter 14 describes the firefly algorithm in depth an has multiple examples of the firefly algorithm implemented. The book cites \cite{FireFly_MGA_Article} . 

\subsubsection{The benefits of cooperation in a highly renewable European electricity network \cite{PypsaModel}}
Article describing simulations using the PyPSA-EUR-30 model. There is a great explanation of the math behind PyPSA 

\subsubsection{The role of spatial scale in joint optimisations of generation and transmission for European highly renewable scenarios\cite{spatialInfluence} }
An article exploring the influence of spatial simplification on energy models. An exapmle using k-means to perform spatial simplification is shown.  

\subsubsection{Modelling to generate alternatives with an energy system optimization model \cite{DECAROLIS2016}}
Another article by DeCariolis exploring the HSJ MGA methodology on energy system optimization 

\clearpage

{\printbibliography[heading=head]
}

\end{document}














%%%%%%%%%%%%%%%%%%%%%%%%%%%%%%%%%%%%%%%%%%%%%%%
%minifigures
\begin{figure}[H]
  \centering
  \begin{minipage}[b]{0.48\textwidth}
    \includegraphics[width=\textwidth]{metalmapel_Cu-K.jpg}
    \caption{Marked Copper in the sample}
  \end{minipage}
  \hfill
  \begin{minipage}[b]{0.48\textwidth}
    \includegraphics[width=\textwidth]{metalcarto.jpg}
    \caption{Marked Copper, Silver and Iron in the sample}
  \end{minipage}
\end{figure}
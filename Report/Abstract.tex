
\chapter{Abstract}

\begin{adjustwidth}{-10pt}{-20pt}



% Problem 
% - CO2 emissions must be reduced to prevent climate change 
% - Energy-economic modeling as tool to decrease CO2 emission 
% - How model optimizations is used today
% - Large uncertainty in model structure and input data = large uncertainty in results
% - HSJ MGA 
To limit the extent of irreversible climate change and accepting public opinion expressed in Greta Thunberg's speech at the UN Climate Action summit, drastic measures are needed to reduce the emission of greenhouse gases. The majority of $\text{CO}_2$ emitted by humans are results of energy production to cover the ever-rising energy demand including transportation, heating, and electricity. To assist scientists and policymakers in their stride to reach ambitious goals in the reduction of CO2 emissions, analysis tools must be developed. An important tool, when it comes to the planning of global and local energy networks are numeric techno-economic energy models. These models are capable of providing great insight into complex systems such as the European electricity grid and allow the user to make predictions about future needs and design strategies. 

Numeric energy-economic models do however suffer from great uncertainties arising from flaws in the mathematical formulation and construction of the energy-economic model referred to as structural uncertainty. An example of flaws in the mathematical formulation could be unmodeled constraints such as public acceptance issues.
If these uncertainties are not addressed, the results become untrustworthy and end up providing little to no insight.
Until recently, no methods for addressing structural uncertainty of the techno-economic models existed. This changed in 2010 when J. DeCarolis published a paper proposing a technique called "Modeling to generate alternatives (MGA)" doing just so. The root cause of structural uncertainty cannot be addressed, as the origin of structural uncertainty is hard to define. Instead, one must investigate all solutions near the one found to be optimal, and estimate the likelihood of these near-optimal solutions being the true optimal solution. 
%In the technique proposed by J. DeCarolis, a finite set of maximally different near-optimal solutions are found. The difference in the found solutions can then be used as a measure of structural uncertainty and provides a variety of alternatives to the optimal configuration of the energy system. 

% Objective 
% - Study the nature of the near optimal feasible space
% - Define MGA approach that will search evenly across the near optimal feasible space
The proposed technique by J. DeCarolis does, however, suffer from a range of flaws, arising from lacking structure in the manner near-optimal solutions are found. To obtain a complete picture of all near-optimal solutions, a structured method of finding these is needed. 
The objective of this thesis is to explore the characteristics of all near-optimal solutions contained within the near-optimal feasible decision space and to develop a new technique that in a structured manner can explore all solutions located within this space.  

% Methodology 

% Results
% - MGA shows as a feasible tool 
% - Multiplicity is wery important to consider 
Analysis of the common mathematical formulation of the numeric techno-economic model reveals that the model consists of linear constraints and therefore, the near-optimal feasible decision space, containing all near-optimal solutions to the model, must be convex. 
Knowing these properties, a technique has been developed capable of searching the entire near-optimal feasible decision space. The technique iteratively converges towards the full solution and provides statistical information about all near-optimal solutions. 
Furthermore, a method reducing the complexity of the mathematical problem, by a grouping of variables is proposed. Grouping the variables in the model to form a new set of variables does however reduce the amount of information obtained by solving this simplified problem. The effects of grouping the model variables are explored, and the effect is found to be significant, but predictable. 
The developed method is applied to a model of the European electricity grid. The usefulness of the technique is proven as it provides information about the distribution of technology capacities in all near-optimal solutions to the used model of the European electricity grid. 


\end{adjustwidth}







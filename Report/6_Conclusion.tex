
\chapter{Conclusion}


% Answer to main resarch question
The near optimal feasible space of techno-economic models, such as the one presented in this project, contains information about all feasible solutions unobtainable through classic optimization. A structured method capable of mapping the entire near optimal feasible space of techno-economic energy models have been developed in this project, building on the principles of MGA approaches. The developed method uses a two step approach, where the hull containing all near optimal solutions are initially found through strategic alterations of the objective function, combined with the introduction of a constraint confining the near optimal feasible space. In the second step of the developed method the hull containing all near optimal solutions is sampled, hereby extracting a dataset representing all near optimal solutions. 


%Summarize and reflect on the research
Initially the mathematical formulation of the techno-economic energy model was investigated, revealing that the near optimal decision space is convex. As the number of decision variables in the presented model is more than half a million a method of reducing the number of variables was investigated. Neglecting hourly dispatch of energy from the individual plants, reduce the number of decision variables included drastically. Neglecting hourly dispatch introduces no significant loss in information as this project focuses on distribution of technology capacities rather than plant operation. Further reduction of the number of decision variables can be achieved by grouping the individual variables into new variables through summation. The selection of grouped variables to include in the studies allows the modeler to focus the studies on areas of interest. An effect of grouping variables, is that the distribution of solutions across the decision space is no longer even. This effect was investigated and results revealed that sample points located close to the center of the decision space of grouped variables, cover over a much larger number of solutions than sample points located on the perimeter of the decision space. The result of this is that the distributions of technology capacity squeeze together, indicating that solutions located near the center of the decision space is much more likely than extreme solutions.

Using the developed MGA method on a techno-economic model of the European electricity grid several experiments was conducted. The experiments successfully extracting data about the techno-economic models near optimal feasible space, revealing large solution flexibility within the near optimal feasible space at a small variance in total system cost. These results emphasize the importance of addressing structural uncertainty in the model through the use of MGA algorithms. From the results, information about technology capacity ranges was extracted, revealing must have technologies as $\text{CO}_2$ emissions are decreased. Using the Gini coefficient as a measure of equality in the location of energy producing technology compared to energy demand, revealed that as $\text{CO}_2$ emissions are decreased larger inequality arises. 

The computing time required by the developed method was estimated, and practical limits identified. Especially the number of variables included in the decision space investigated by the method was found to have a large effect on required computing time. A successful study including seven variables in the decision space was conducted however, the practical limit to the number of decision variables have been estimated to be ten. Within these limitations the model has proved very useful, and through intelligent selection of decision variables a wide range of studies can be performed. 
%Make recommendations for future work on the topic
Future studies could include the investigation of models implementing storage technologies or models implementing larger selections of renewable energy technologies. 
Overall, the developed method has been found to provide much greater insights about the near optimal solution space of techno-economic models, compared to previous MGA methods, and the use of the developed MGA method in future work will contribute to a greater understanding of energy systems and the available possibilities when conducting energy system synthesis.
%Show what new knowledge you have contributed



%Clearly state the answer to the main research question
%Summarize and reflect on the research
% - Studies has been conducted revealing large flexibility 
% - Multiplicity has been proven
% - Performance of MGA algorithm
% - Comparison to previous MGA solutins
%Make recommendations for future work on the topic
%Show what new knowledge you have contributed

% Checklist
%- The main research question has been concisely answered.
%- The overall argument has been summarized.
%- There is reflection on the aims, methods and results of the research.
%- Any important limitations have been mentioned.
%- Relevant recommendations have been discussed.
%- The contributions of the research have been clearly explained.

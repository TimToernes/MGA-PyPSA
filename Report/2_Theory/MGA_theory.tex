\subsection{Modelling to Generate Alternative (MGA)}\label{sec:MGA}
In this section the basic principles of MGA will be explained together with the benefits and challenges this technique introduces. 

\subsubsection{Motivation for using MGA}

In the field of mathematical modeling, the scientist aim to produce models representing physical systems as realistically as possible. However, some degree of uncertainty in the models is inevitable as model fidelity is limited by a range of factors including: numeric precision, uncertainty of data, model resolution etc. Modeling of energy systems is a field especially prone to large model uncertainties, deriving not only from lack of fidelity, but from factors such as unmodeled objectives and structural uncertainty \cite{DeCarolis_MGA}. 

The MGA approach was first introduced in 1982 by Brill et al. \cite{Brill_MGA_1982}, in the field of operations research/management science. This is a field where unmodeled objectives and structural uncertainty. 


The basic insight can be
summarized as follows: Because it is not possible to develop a complete
mathematical representation of complex public planning problems,
structural uncertainty in optimization models will always exist. As a
result, the ideal solution is more likely to be located within the model's
inferior region rather than at a single optimal point or along the noninferior frontier (Brill, 1979)

Policy makers often have strong concerns outside the scope of most models
(e.g., political feasibility, permitting and regulation, and timing of
action), which implies that feasible, suboptimal solutions may be
preferable for reasons that are difficult to quantify in energy economy
optimization models.

The purpose of MGA is to efficiently search the feasible
region surrounding the optimal solution to generate alternative
solutions that are maximally different. 

\subsubsection{Technical explanation of the optimization problem}


The optimization problem at hand is a simplified energy economic model of Europe, build with focus on exploring the composition of VRES (variable renewable energy sources) on a global and national scale. In the model each country is represented as a note connected to the surrounding countries through a link. Each country has three energy producing technologies available, gas, wind and solar power. A data resolution of 1 hour is used, and simulations run over an entire year. 

Analyzing this model one finds that the following variables are relevant for the optimization problem:

\begin{itemize}
    \item Hourly dispatch of energy from the given plants in the given countries $P_i$.
    \item Total installed capacity of the given technologies in the given countries $P_{nom_i}$
    \item Hourly power flow in each line connecting two countries
    \item Total install line capacity for all lines $S_{nom_i}$

\end{itemize}

The objective function for the optimization problem then becomes: 

\begin{equation}
    p = \sum P_i \cdot m_{p_i} + \sum P_{nom i} c_i + \sum S_{nom} \cdot c_s 
\end{equation}{}

Subject to the constraints 


\subsubsection{Technical explanation of MGA HSJ}

\subsubsection{Other MGA approaches}

\subsection{Novel MGA approach}
In this section a novel approach towards MGA optimization of energy networks will be presented. Based on the same concepts as presented in \ref{sec:MGA} this method seeks to explore not only a few alternative solutions from the decision space, but the entire decision space. Hereby an in depth knowledge of the possible solution is obtained providing insight in the distribution of alternative solutions. \\

Analyzing the original energy network optimization problem it is clear that this the constraints in the system defines an open convex set, as nothing prevents to model from installing excessive amounts of energy sources, however the objective function will seek to minimize installed capacities end hereby cost. However, a lower bound for installed capacities is present as energy demand must be met for every hour. 
Introducing the MGA constraint from equation \ref{eq:MGA_constraint} this open set is closed, as the installed capacities is limited in size by the limited maximum cost of the system. 

\begin{equation}\label{eq:MGA_constraint}
	f(\vec{x}) \leqslant f(\vec{x}^*) \cdot (1+\epsilon)
\end{equation}

As we now have a closet set that must be convex since only linear constraints is used to define it, it now is possible to explore the shape of this convex set. Assuming that all constraints used including the MGA constraint is linear, the convex set must be a polyhedral and therefore it is possible to define the shape of this set with a finite number of vertexes. \\

However, finding these vertices is no trivial task. The method proposed here will use the following steps to approximately find all vertices. 1) maximize and minimize all variables. 2) Based on these points define a convex hull. 3) change objective function to search in the direction of the normal of each face on the hull. 4) Update hull with new points and repeat 3 and 4 until hull volume stops increasing. \\

Pseudo code:

\begin{itemize}[label={}]
	\item Solve network subject to regular constraints and with original objective function
	\item Add MGA constraint !Equation number
	\item while $\epsilon>tol$
	\begin{itemize}[label={}]
		\item If first loop
		\begin{itemize}[label={}]
			\item directions = max and min all variables
		\end{itemize}
		\item Else
		\begin{itemize}[label={}]
			\item directions = normals to hull faces
		\end{itemize}
		\item for direction in directions
		\begin{itemize}[label={}]
			\item objective function = direction[i] * variable[i]
			\item point on convex hull += solve problem subject to objective function
		\end{itemize}
		\item hull = ConvexHull ( points on convex hull)
		\item $epsilon$ = new hull volume - old hull volume / hull volume
	\end{itemize}
	\item Evenly distribute points in hull 
	\item Plot histogram using evenly distributed points. 
\end{itemize}















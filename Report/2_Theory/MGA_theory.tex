
\section{Energy Economic model of Europe}
In this section the composition of the energy-economic model used in this project will be described. 

\subsection{Topology}
The model used in  this project is based on the work presented in \cite{PyPSA_euro_30_model}, where a model spanning the electricity grid of 30 European countries is formulated as a techno-economic linear optimization problem. Countries included in the model are the EU-28 countries not including Cyprus and Malta, instead including Norway, Switzerland, Serbia and Bosnia and Herzegovina.

The topology of the network, presented on \ref{fig:network_lay}, is such that each node represents a country and the links represent international HVDC or HVAC links. The links included are based on currently installed international transmission lines. 

	
\begin{figure}[H]\centering
	\includegraphics[width=0.75\textwidth]{./Images/network_layout}
	\caption{Network layout}
	\label{fig:network_lay}
\end{figure}


All model input parameters are based on 2011 values as this is the earliest year with all data available. The temporal resolution of the model is hourly, with all simulations spanning a full year. Technology costs are all valued in 2011 Euros. 
Should also be included:
<
\subsection{Energy production}

Each node in the network, has energy producing technologies available, with initial capacities being zero. The available energy producing technologies used in this project is: Onshore wind, offshore wind, Solar PV and OCGT. In the model all technology capacities are expandable limited only be the geographical potential. 

The geographical potentials used are calculated following the work of  \cite{PyPSA_euro_30_model}. In the calculation of geographical potential, the potential available area suited for either onshore wind, offshore wind and solar PV, must first be defined. These areas was found by allowing certain technologies to be installed only in areas with certain land use types. Hereby restricting onshore wind farms from being installed in cities and solar PV plants to be installed in forests etc. The placement of offshore wind farms was restricted to areas with a water depth of less than 50m. Furthermore, all nature reserves was excluded from the potential areas. As competing land use and likely public acceptance issues will occur, the found potential areas are set to bee only 20\% of the found area for onshore and offshore wind and only 1\% for solar PV. 
Assuming a maximum nominal installation density of 10 $MW/km^2$ for offshore and onshore wind power, and 145 $MW/km^2$ for solar PV, it is possible to calculate the geographical potential for the three technologies all across Europe. Geographical potential for the three technologies are presented on figure \ref{fig:geographical_potential}.

\begin{figure}[H]\centering
	\includegraphics[width=0.95\textwidth]{./Images/geographical_potential}
	\caption{!!! PLACE HOLDER !!! Geographical potential GW/km2}
	\label{fig:geographical_potential}
\end{figure}

The hourly energy production of all variable renewable energy sources is limited by the production potential given by the weather. Following \cite{PyPSA_euro_30_model}, the availability was calculated using historic weather data for 2011 from \cite{ClimateForecastSystem} with a spatial resolution of 40x40 km and hourly temporal resolution. The weather data is first converted to generation potentials for each 40x40 km cell using the REatlas software \cite{ANDRESEN20151074}, and then the national means are found. 


\begin{figure}[H]\centering
	\includegraphics[width=0.95\textwidth]{./Images/geographical_potential}
	\caption{!!! PLACE HOLDER !!! Mean wind generation potential}
	\label{fig:generation_potential}
\end{figure}

The dispatchable energy sources available in all countries are chosen to be open cycle gas turbines (OCGT), as they have a high flexibility and good load following capabilities, therefore making them suitable as a backup generator in a highly decarbonized scenario. They do however not necessarily produce realistic results, when used in scenarios with low decarbonization. The capacities and energy generation of the gas turbines are contrary to the variable renewable energy sources, not limited by geographical or generation potentials. They are however, limited by the maximum allowable CO2 emission. The CO2 emission intensity of the open cycle gas turbine is 0.19 t/MW.


In all simulations the capacities of all energy generators are initially set to be zero, with the capability to be expanded until geographical potentials or CO2 limits further expansion. The cost of expanding capacities is calculated as annualized cost, given as the annualized investment cost plus fixed annual operations and maintenance cost. Annualized investment cost is calculated by multiplying the annuity factor \ref{eq:annuity} by the investment cost. 

\begin{equation}\label{eq:annuity}
a = \frac{r}{1 - \frac{1}{(1+r)^n}}
\end{equation}

Where $r$ is the discount rate, and $n$ is the expected lifetime of the given technology. In this project a discount rate of 7\% is used. The lifetime of the individual technologies are listed in table \ref{tab:cost_data}. All cost data are based on the 2030 values presented in \cite{Schroder2013Current}.

\begin{table}[]
	\begin{tabular}{l|llll}
		Technology      & \makecell[c]{Investement \\ {[}€/MW{]}}	& \makecell[c]{Fixed O\&M \\ {[}€/kW/year{]}} & \makecell[c]{Marginal cost \\ {[}€/MWh{]}}	& \makecell[c]{lifetime \\ {[}years{]}} \\ \hline
		Onshore Wind    &       1182  		&   35      &   0.015       &   25       \\
		Offshore Wind	&		2506		&	80		&	0.02		&	25		\\
		Solar PV   		&       600    		&   25      &   0.01    	&   25       \\
		OCGT       		&       400    		&   15      &   58.4        &   30      \\
		Transmission	& 400 €/MW km +150000 pr line & 2\% & 0 		&   40 
	\end{tabular}
	\caption{Generator parameters are based on the values from \cite{Schroder2013Current}, and transmission parameters are based on the work presented in \cite{HAGSPIEL2014654}.}
	\label{tab:cost_data}
\end{table}

\subsection{Energy demand}
The data for the hourly electricity demand found in the European Network of Transmission System Operators data portal is used as energy demand \cite{ENTSO-E}. The data has a resolution of one hour, and is provided for all countries included in the model. 

\subsection{Energy transmission}
In the model used in this project, all transmission lines are treated as transport models with a coupled source and sink, only constrained by energy conservation at each connecting node. Transmission loss is thereby not considered. This approximation is assumed to be acceptable as most international transmission lines already are, or probably will be in the near future, controllable point-to-point high voltage direct current (HVDC) lines. 

Line capacities initially start as zero, and can then be expanded if found feasible in the optimization, with no constraint on the maximum allowable capacity. The investment cost of line capacity is calculated as a cost pr MWkm plus an additional cost for a high voltage AC to DC converter pair. The price of a high voltage AC to DC converter pair is set to be 150000€ regardless of line capacity \cite{HAGSPIEL2014654}. 

The length of each line is set as the distance between the centroids of each connecting country plus an additional 25\%. The extra 25\% is added to the line length as competitve land use and public acceptance issues will prohibit lines from being placed in optimal positions. 

Furthermore, to satisfy n-1 security the price is adjusted with a factor of $1.5$, to account for the extra installed capacity needed, as shown in \cite{PyPSA_euro_30_model}. 

\begin{equation}
c_l = \left( L*I_s*1.25+150000 \right) *1.5*1.02*a
\end{equation}

1.25 = 25\% extra length due to land use competition
150000 = Price of DC converter pair
1.5 = n-1 security 
1.02 = 2\% FOM (fixed operations and maintanance cost)
a = annuity 



	
\fxnote{Consider showing some time series for demand etc}
	
	
	
	
	
\section{Mathematical formulation}\label{sec:OptimizationProblem}


The optimization problem at hand is a simplified energy economic model of Europe, build with focus on exploring the composition of VRES (variable renewable energy sources) on a global and national scale. In the model each country is represented as a note connected to the surrounding countries through a link. Each country has three energy producing technologies available, gas, wind and solar power. A data resolution of 1 hour is used, and simulations run over an entire year. 

Following the naming convention from \cite{PyPSA_euro_30_model}, indexing the notes in the network with the variable $n$, the power generating technologies by $s$, the hours in the year by $t$ and the possible connecting power lines by $l$, the contributing variables to the objective function describing the total annualized system cost is the following: 

\begin{itemize}
	\item Hourly dispatch of energy from the given plants in the given countries $g_{n,s,t}$ with the marginal cost $o_{n,s}$.
	\item Total installed capacity of the given technologies in the given countries $G_{n,s}$ with the capital cost $c_{n,s}$.
	\item Total installed transmission capacity for all lines $F_{l}$ with the fixed annualized cost $c_{l}$.
	
\end{itemize}

The objective function for the optimization problem then becomes: 

\begin{equation}
min \; p = \left( \sum_{n,s} c_{n,s} G_{n,s} + \sum_l c_l F_l + \sum_{n,s,t} o_{n,s} g_{n,s,t} \right)
\end{equation}{}

This objective function is subject to a range of constraints ensuring realistic behavior of the system. As described in \cite{PyPSA_euro_30_model} a power balance constraint is issued to ensure stable operation of the network. These constraints force the sum of energy produced and consumed in every hour to equal zero. The hourly electricity demand at each node is described by $d_{n,t}$, the incidence matrix describing the line connections is given by $K_{n,l}$ and the hourly transmission in each line is described as $f_{l,t}$. Then the power balance constraint becomes:

\begin{equation}
\sum_s g_{n,s,t} - d_{n,t} = \sum_l K_{n,l} f_{l,t} \; \forall n,t
\end{equation}

For all conventional generators the maximum hourly dispatch of energy is limited by the installed capacity. It is important to node that for all simulations performed in this project the installed capacity is a variable. 

\begin{equation}
0\leq g_{n,s,t} \leq G_{n,s} \; \forall n,s,t
\end{equation}

The dispatch of variable renewable energy sources (wind and solar) is not only limited by the installed capacity, as availability, hence the name, is variable. Therefore the constraint for dispatch of variable renewable energy sources become:

\begin{equation}
0 \leq g_{n,s,t} \leq \overline{g}_{n,s,t} G_{n,s} \; \forall n,s,t
\end{equation}

Where $\overline{g}_{n,s,t}$ represents the normalized availability per unit capacity. 

The installed capacity is constrained by the geographical potential calculated in \cite{PyPSA_euro_30_model}.

\begin{equation}
0 \leq G_{n,s} \leq G_{n,s}^{max} \; \forall n,s
\end{equation}

All transmission lines in the model modelled with a controllable dispatch constrained by the fact that there must be energy conservation at each node the line is connected to. !! Something here about which lines is included !!!! . Furthermore the transmission in each line is limited by the installed transmission capacity in each line. 

\begin{equation}
|f_{l,t}| \leq F_l \; \forall l,t
\end{equation}

In the model it is possible to activate a CO2 constraint, limiting the allowed CO2 emissions for the entire energy network. As in \cite{PyPSA_euro_30_model} the constraint is implemented using the specific emissions $e_s$ in CO2-tonne-per-MWh of the fuel for each generator type $s$, with the efficiency $\eta_s$ and the CO2 limit $CAP_{CO_2}$. 

\begin{equation}
\sum_{n,s,t} \frac{1}{\eta_s}g_{n,s,t} e_s \leq CAP_{CO_2}
\end{equation}

The model is implemented in the open source software PyPSA \cite{Pypsa}, using much of the software presented in \cite{PyPSA_euro_30_model}. Optimization of the model is performed with the optimization software Gurobi \cite{Gurobi}. 

\section{Properties of the near optimal feasible space}\label{sec:properties_of_hull}

Analyzing the original optimization problem one can deduct that the set including all feasible solutions $W$, given by equation \ref{eq:feasible_space}, must be convex, as all constraints $f_i$ and the objective function $f_0$, are linear and therefore satisfy equation \vref{eq:convex_requirement}, thus ensuring convexity \cite{ConvexOpimization}. 

\begin{equation}\label{eq:convex_requirement}
f_i(\alpha x + \beta y) \leq \alpha f_i(x) + \beta f_i(y) \; \forall \; x, y \in \mathbb{R}^d and  \; \alpha, \beta \in \mathbb{R}
\end{equation}

Furthermore, when all variables are bounded; hourly production by the power balance constraint and installed capacity by geographical potential and CO2 emission constraints, the feasible decision space is not only convex but also closed. If the geographical potential constraint, or the CO2 emission limit is excluded the feasible decision space becomes an open convex space as illustrated on \vref{fig:sketch_feasable_space}, this does however not have any immediate consequences, as the objective function increases as one moves in the open direction of the space. 

\begin{figure}[ht]
	\centering
	\incfig{Feasible-space}
	\caption{A sketch of a one dimensional feasible space with MGA constraint }
	\label{fig:sketch_feasable_space}
\end{figure}

The set containing the feasible decision space can be described with equation \ref{eq:feasible_space}. 

\begin{equation}\label{eq:feasible_space}
W = \{ \vec{x}\in \mathbb{R}^d | f_i(\vec{x}) \geq 0 \}
\end{equation}
!!! Im not 100\% on this !!! 

Where the vector $\vec{x}$ is containing all variables in the model, and $ f_i(x) \geq 0 $ denotes that all constraints must be satisfied. 

It is important to note that the variables $\vec{x}$ that defines the decision space variables in the original problem include all hourly technology dispatches $g$, all installed technology capacities $G$ and all installed line capacities $F$. 

\begin{equation}
\vec{x} = \{g_{n,s,t} \wedge G_{n,s} \wedge F_l \; \forall \; n,s,t,l \}
\end{equation}


Therefore, the dimensionality of the decision space must be given by the number of individual dispatch decisions given by $n\cdot s \cdot t$ plus the number of capacities to optimize, given by $n\cdot s$ plus the number of line capacities to optimize $l$. The dimensionality of the full problem is therefore given by equation \vref{eq:dimentionality}.

\begin{equation}\label{eq:dimentionality}
d = n\cdot s \cdot t + n\cdot s + l
\end{equation}


In the case of the reference model used in this project that gives 
$ 30 \cdot 3 \cdot 8765 + 30 \cdot 3 + 52 = 788992 $ 


The true dimensionality might be lower, as some variables do have strong corelations. 

% Number of faccets on n dim simplices

\subsection{Dimensionality reduction}

As the dimensionality of the decision space is very large, and therefore becomes very unhandy to work with, it makes sense to look at a subspace of lower dimensionality. One could choose to ignore the hourly dispatch of energy from the individual generators, hereby reducing the dimensionaly by a substantial amount. 

\begin{equation}
d^* = n\cdot s + l
\end{equation}

In that case the dimensionality would only be $d^* = 30\cdot 3 + 52 = 142$. 

Therefore, we design a new set $W^* \subseteq \mathbb{R}^{d^*}$, as a subset of $W$, but with reduced dimensionality, as it is only including all installed capacities as variables. 

\begin{equation}
	W^* = \{ x^* \in x | x \in W \}
\end{equation}

Because the set $W^*$ is a subset of $W$, any solution $\vec{x^*}$ found to be inside $W^*$, must also lie inside the original feasible set $W$. But not necessarily vice versa. 

By decreasing the dimensionality in this manner all information about hourly plant operation is lost, but as the focus of this project is to analyze the distribution of capacities, this is not a major loss. The loss in information is greatly overcome by the ease of computation. 

\subsubsection{Further reduction by grouping of variables}
If desired it is possible to further reduce dimensionality, by grouping variables, on behalf of further loss of information. An example would be to group all individual technologies in groups containing all installed capacity of that given technology across the entire network. This new set would then have a dimentionality of:

\begin{equation} \label{eq:dim_d**}
d^{**} = s
\end{equation}

Which in this project is only $3$. This is now a very manageable dimension size. The new set $W^{**} \subseteq \mathbb{R}^{d^{**}}$ including only summed capacity sizes for all technologies can be designed as:

\begin{equation}
	W^{**} = \{ x^{**}  | \sum_n G_{n,s} \; \forall \; s  \}
\end{equation}

%\begin{equation}\label{eq:variable_x**}
%x^{**} = \{\sum_n G_{n,s} \; \forall \; s  \}
%\end{equation} 

Any solution $x^{**}$ found in $W^{**}$, must satisfy the constraints $f_{\vec{x}} \leq 0$ and therefore the set $W^{**}$ is a subset of $W$. This means that a solution found in $W^{**}$ lies inside $W$.

%\begin{equation}
%W^{**} = \{\vec{x}^{**} \in \mathbb{R}^{d^{**}} |    \}
%\end{equation}

\section{Numeric optimization}



\section{Modeling to Generate Alternatives (MGA)}\label{sec:MGA}
In this section the basic principles of MGA will be explained together with the benefits and challenges this technique introduces. 

\subsection{Motivation for using MGA}

In the field of mathematical modeling, the scientist aim to produce models representing physical systems as realistically as possible. However, some degree of uncertainty in the models is inevitable as model fidelity is limited by a range of factors including: numeric precision, uncertainty of data, model resolution etc. Modeling of energy systems is a field especially prone to large model uncertainties, deriving not only from lack of fidelity, but from factors such as unmodeled objectives and structural uncertainty \cite{DeCarolis_MGA}. 

The MGA approach was first introduced in 1982 by Brill et al. \cite{Brill_MGA_1982}, in the field of operations research/management science. This is a field where unmodeled objectives and structural uncertainty, are highly influential. 

!! CITATION !!
The basic insight can be
summarized as follows: Because it is not possible to develop a complete
mathematical representation of complex public planning problems,
structural uncertainty in optimization models will always exist. As a
result, the ideal solution is more likely to be located within the model's
inferior region rather than at a single optimal point or along the noninferior frontier (Brill, 1979)

Policy makers often have strong concerns outside the scope of most models
(e.g., political feasibility, permitting and regulation, and timing of
action), which implies that feasible, suboptimal solutions may be
preferable for reasons that are difficult to quantify in energy economy
optimization models.

The purpose of MGA is to efficiently search the feasible
region surrounding the optimal solution to generate alternative
solutions that are maximally different. !!!



\subsection{Technical explanation of MGA HSJ}

The MGA technique was first introduced in 1982 by Brill et. al in the article \cite{Brill_MGA_1982} and later rediscovered by DeCarolis in \cite{DeCarolis_MGA} for use in energy system optimization. The tecnique lets the user search the near optimal feasible decision space for an optimization problem such as the one addressed in this project described in \ref{sec:OptimizationProblem}. 

In section \ref{sec:OptimizationProblem} a series of constraints bounding the network model is listed. Together these constraints form a feasible region that can be described as a convex set in a $d$ dimensional space. Where d is the number of variables in the model. The feasible set is convex as all bounding constraints are linear. The fact that linear constraints form a convex set is shown in \cite{ConvexOpimization}. The MGA technique introduces yet another constraint limiting the size of this convex set even further by limiting the objective function value of all feasible points to be within a certain range of the optimal solution. The goal of the MGA technique is to explore a finite set of alternative solutions located within this convex set. 

In the orginal articel by Brill et. al \cite{Brill_MGA_1982} the HSJ MGA technique is descrbed with the following steps. 

(1) obtain an initial optimal solution for the problem at hand; (2) define a target value for the objective function by adding a user specified amount of slack to the value of the objective function in the initial solution (3) introduce the constraint limiting the objective function to surpass this target value, to the model (4) formulate a new objective function that seeks to minimize the sum of decision variables that had non zero values in the previous solution of the problem (5) iterate the reformulated problem, updating the objective function every time (6) terminate the optimization when the new solution is similar to or close to any previously found solution. Step 3 and 4 was described mathematically in \cite{Brill_MGA_1982} as follows:

\begin{equation}
\begin{split}
Minimize \; &  p = \sum_{k \in K} x_k \\
Subject to \; &  f_j(\vec{x}) \leq T_j \; \forall \;  \vec{x}\in X
\end{split}
\end{equation}

In this formulation $k$ represents the variable indices for the variables with nonzero values in the previous solution, $j$ is the objective function indices if multiple objective functions exists, $f_j(\vec{x})$ is the evaluation of the $j$'th objective function and $T_j$ is the target value specified for the particular objective function. In the formulation of the constraint $\vec{x}\in X$ specifies that all previously defined constraints still applies as all new solutions $\vec{x}$ must be a part of the set of feasible solution vectors from the original formulation $X$.

How the new objective function precisely is formulated and which variables to include is discussed in \cite{DECAROLIS2016}, where two alternative approaches of defining the new objective function is presented. One approach suggest giving all nonzero variables from the last iteration a weight of 1 in the new objective function. This approach does not consider weight from previous iterations. However, the second approach suggests adding on to the coefficient with a factor of +1 for every time one variable has appeared with nonzero in a row, hereby further increasing the intended to reduce the use of that specific technology. This 

\subsection{Other MGA approaches}



\section{Novel MGA approach}\label{sec:Novel_MGA}

In this section a novel approach towards MGA optimization of energy networks will be presented. Based on the same concepts as presented in \ref{sec:MGA} this method seeks to explore not only a few alternative solutions from the decision space, but the entire decision space. Hereby an in depth knowledge of the possible solution is obtained providing insight in the distribution of alternative solutions.

An important feature about the method developed is that it can be used for any dimensional decision space. 

The method developed can be divided into two phases. In the first phase, the shape of the feasible near optimal decision space is found, and in the second phase relevant data is extracted from the found space. 

\subsection{Decision space mapping}
As explained in section \vref{sec:properties_of_hull}, the near optimal feasible space will always be convex, and can either be closed or not. However, when the MGA constraint from equation \vref{eq:MGA_constraint} is introduced the space will be closed. 

\begin{equation}\label{eq:MGA_constraint}
f(\vec{x}) \leqslant f(\vec{x}^*) \cdot (1+\epsilon)
\end{equation}

As we now have a closed convex space, it now is possible to explore the shape of this convex set. Assuming that all constraints used including the MGA constraint is linear, the convex set must be a polyhedral and therefore it is possible to define the shape of this set with a finite number of vertexes. !!! This might not be the case for CO2 constraint!!!! \\

However, finding these vertices is no trivial task. In the method developed, all solutions found, that lie within the near optimal feasible space is treated as a point in that space. Furthermore, the possibility of letting the objective function search in a given direction in the decision space is utilized, by replacing the original objective function to an objective function on the from presented in \vref{eq:objective_func_face_normal}.

\begin{equation}\label{eq:objective_func_face_normal}
Minimize \; p = \vec{n}_i\vec{x}
\end{equation}

Where $\vec{n}_i$ is the $i$'th normal vector. 


The method proposed here will use the following steps to approximately find all vertices. 

\begin{enumerate}
	\item Find initial solution
	\item Add MGA constraint
	\item Maximize and minimize all variables
	\item Based on these points define a convex hull, and define all face normals
	\item Iterate over each face normal and change objective function to \vref{eq:objective_func_face_normal}
	\item Add the newly found points to list of points and define new hull and its face normals 
	\item Repeat step 5 and 6 until the size of the convex hull converges 
\end{enumerate}

The convergence criteria used in this project is that the hull size must not increase by more than 2\% in two consecutive iterations. 

The result of following these steps is a list of points defining a hull in $d$ dimensional space, however this on its own does not provide much usefull information. To gain any knowledge about the network being analysed on must follow the steps provided in part two of this method. 

\subsection{Hull fill}










\subsection{Pseudo code}

\begin{itemize}[label={}]
	\item Solve network subject to regular constraints and with original objective function
	\item Add MGA constraint !Equation number
	\item while $\epsilon>tol$
	\begin{itemize}[label={}]
		\item If first loop
		\begin{itemize}[label={}]
			\item directions = max and min all variables
		\end{itemize}
		\item Else
		\begin{itemize}[label={}]
			\item directions = normals to hull faces
		\end{itemize}
		\item for direction in directions
		\begin{itemize}[label={}]
			\item objective function = direction[i] * variable[i]
			\item point on convex hull += solve problem subject to objective function
		\end{itemize}
		\item hull = ConvexHull ( points on convex hull)
		\item $epsilon$ = new hull volume - old hull volume / hull volume
	\end{itemize}
	\item Evenly distribute points in hull 
	\item Plot histogram using evenly distributed points. 
\end{itemize}


\begin{figure}[H]\centering
	\includegraphics[width=0.95\textwidth]{./Images/step1}
	\caption{Geographical potential GW/km2}
	\label{fig:step1}
\end{figure}

\begin{figure}[H]\centering
	\includegraphics[width=0.95\textwidth]{./Images/step2}
	\caption{Geographical potential GW/km2}
	\label{fig:step2}
\end{figure}

\begin{figure}[H]\centering
	\includegraphics[width=0.95\textwidth]{./Images/step3}
	\caption{Geographical potential GW/km2}
	\label{fig:step3}
\end{figure}

\fxnote{make these plots 2D instead}

\section{Gini coefficient}

In this project, the gini coefficient is used to express the equality in the distribution of energy generation versus consumption. The gini coeficient is calculated as the relationship between the area under the total equality line and the area between the equality line and the Lorentz curve. \fxnote{Reformulate this about the gini coefficient}

Therefore, a scenario where every country over the duration of an entire year, produces as much energy as it consumes, would have a gini coefficient of 0, and represent the equality line on figure \ref{fig:Gini}. A scenario where one country is producing all energy and consuming none, would on the other hand have a gini coefficient of 1, and represent total inequality. 

\begin{figure}[h]\centering
	\includegraphics[width=1.\textwidth]{./Images/optimal_solutions_gini}
	\caption{Lorentz curves for the four optimal solutions \fxnote{Consider if this plot should stay}}
	\label{fig:Gini}
\end{figure}


\section{Implementation and utilization of parallel programming}
As the MGA approach described in section \ref{sec:Novel_MGA} requires a high number of similar optimizations to be performed only with slighly changed objective functions, it is possible to achieve a great performance boost, by utilizing parallel programming. 




\section{Experiment design}
In this section all performed experiments will be explained. 

\subsection{4D experiment}

\subsection{CO2 experiment}

\subsection{Multiplicity experiment}

\subsection{Spatial grouping experiment}











\chapter*{Notes on references}

\subsubsection*{Impact of CO2 prices on the design of a highly decarbonized coupled
electricity and heating system in Europe\cite{PypsaModel}}
An investigation on the CO2 price levels needed to reduce CO2 emissions. In the article a PyPSA model of Europe is presented. The model could be used in this project. 

\subsubsection*{MODELING TO GENERATE ALTERNATIVES: THE HSJ
APPROACH AND AN ILLUSTRATION USING A
PROBLEM IN LAND USE PLANNING \cite{Brill_MGA_1982}}

This is the original article, \cite{Brill_MGA_1982}, explaining the thoughts behind MGA. In this article the HSJ (Hop Skip Jump) approach is implemented. This article seams to be the mother of all other MGA articles. 

\subsubsection*{MGA: a decision support system for complex, incompletely defined problems\cite{Brill_MGA_1990}}
Elaborating on the MGA approach presented in \cite{Brill_MGA_1982}, and evaluating the performance of MGA as a whole. 

\subsubsection*{Using modeling to generate alternatives (MGA) to expand our thinking on
energy futures\cite{DeCarolis_MGA}}

\cite{DeCarolis_MGA} is one of the first implementations of MGA on energy planning. Uses the HSJ method from \cite{Brill_MGA_1982}.

\subsubsection*{Modeling to generate alternatives: A technique to explore uncertainty
in energy-environment-economy models \cite{MGA}}

In this article MGA is used to explore near optimal solutions in energy network optimization, much like \cite{DeCarolis_MGA}. However a slightly more advanced MGA objective function is used. The objective function to be maximized is the Manhattan distance between the current and all preveiously generated MGA solutions. 

\subsubsection*{Ensuring diversity of national energy scenarios: Bottom-up energy
system model with Modeling to Generate Alternatives \cite{BERNTSEN2017886}}
A different approach towards implementing MGA on energy system planning. Here they use the  EXPANSE software/model to implement MGA on. They use a sort of random search MGA approach.

\subsubsection*{Simulation-Optimization techniques formodelling to generate alternatives in waste management planning \cite{Yavuz2011}}
This article describes the MGA method used in \cite{BERNTSEN2017886}. Here a random population is created and is sorted through a number of itterations. 

\subsubsection*{GENETIC ALGORITHM APPROACHES FOR ADDRESSING UNMODELED OBJECTIVES IN OPTIMIZATION PROBLEMS \cite{Genetic_Algorithms_for_MGA}}
This article describes the basic theory of MGA very well, and introduces two new genetic algorithms, that could be used for MGA. The Algorithms are based on genetic nieching/sharing algorithms. 

\subsubsection*{A Co-evolutionary, Nature-Inspired Algorithm for the Concurrent Generation of Alternatives \cite{FireFly_MGA_Article}}

The article \cite{FireFly_MGA_Article} describes an implementation of the genetic firefly algorithm used to perform MGA. 

\subsubsection*{Swarm Intelligence and Bio-Inspired Computation : Theory and Applications - Chapter 14 \cite{Bio_computation_book}}

The book \cite{Bio_computation_book} Chapter 14 describes the firefly algorithm in depth an has multiple examples of the firefly algorithm implemented. The book cites \cite{FireFly_MGA_Article} . 

\subsubsection*{The benefits of cooperation in a highly renewable European electricity network \cite{PypsaModel}}
Article describing simulations using the PyPSA-EUR-30 model. There is a great explanation of the math behind PyPSA 

\subsubsection*{Transmission needs across a fully renewable European power system}
\cite{RODRIGUEZ2014467}
Article exploring the effect of transmission across the EURO-30 model. 

\subsubsection*{Validation of Danish wind time series from a new global renewable energy atlas for energy system analysis}
\cite{ANDRESEN20151074} REAtlas software

\subsubsection*{The NCEP Climate Forecast System Version 2}
\cite{ClimateForecastSystem} Weather data

\subsubsection*{The role of spatial scale in joint optimisations of generation and transmission for European highly renewable scenarios\cite{spatialInfluence} }
An article exploring the influence of spatial simplification on energy models. An exapmle using k-means to perform spatial simplification is shown.  

\subsubsection*{Modelling to generate alternatives with an energy system optimization model \cite{DECAROLIS2016}}
Another article by DeCariolis exploring the HSJ MGA methodology on energy system optimization 

\subsubsection*{The optimum is not enough: A near-optimal solution paradigm for energy systems synthesis \cite{Optimum_not_enough}}
A different approach for exploring the near optimal feasable space, using a technique that is not quite MGA but very similar. The approach generates a finite set of alternative solutions. 


\subsubsection*{Current and prospective costs of electricity generation until 2050}
\cite{Schroder2013Current} includes cost data for all energy technologies relevant for this study

\subsubsection*{Optimal Combination of Storage and Balancing in a 100\% Renewable European Power System}
\cite{rasmussen2011a}
Article where the optimal mix between wind and solar energy is explored. 


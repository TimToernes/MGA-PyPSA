
\chapter{Introduction}

%\subsection{Problem definition}

% Large Scale Background
High global ambitions for decreased CO2 emission and the resulting increase in implementation of renewable energy sources, introduce higher demands to the energy grid than ever. The volatile nature of renewable energy sources, implemented to reach ambitious CO2 emission goals, drives the need for collaboration/coupling between countries, energy sectors, and energy sources, to handle peak loads and hours of energy scarcity. This complicates the already complex task of energy system synthesis even further, hereby requiring decision makers to have greater in depth knowledge, in a world where rapid decisions and superficial political decisions are becoming more widespread. Therefore, the need for analysis tools providing insights in the constraints and possibilities decision makers must deal with, has never been more present.  

% Narrow background
A frequently used tool to gain insight in the future energy grid compositions, is energy-economic models on either regional, national or international scale. These models can be used to study the behavior and composition of existing and future energy networks, together with the impact of new technologies or structural changes in the networks \cite{Gorm_impact_of_CO2_PYPSA} !!CITE OTHEER WORKS USING energy-economic models!!. 
However, these models do suffer from large uncertainties and the lack of validation possibilities, resulting in unreliable and therefore less informative results. 

Model uncertainty can be categorized as either parametric uncertainty, arising from uncertainty in input parameters and data, or as structural uncertainty introduced by an incomplete or faulty mathematical description of the problem at hand \cite{DeCarolis_MGA} . Structural uncertainty is however not caused by the modelers lack of mathematical talent, but is the result of dealing with a very complex problem, influenced by multiple actors such as policymakers and private company's in the energy sector. 

% Literature review/Current standpoint
Recently an approach for extracting more relevant and less uncertain data from energy-economic models has been proposes by \citeauthor{DeCarolis_MGA}, where a technique called Modeling to Generate Alternatives (MGA), from the field of management research/planning science \cite{Brill_MGA_1982}, is applied to the field of energy planning. MGA allows the modeler to explore the near optimal feasible decision space of the energy-economic model and hereby exploring possible optimal solutions otherwise not found due to structural and parametric uncertainty. The concept of using MGA algorithms on energy planning problems have been further studied and the result presented in a range of articles and papers; \cite{DECAROLIS2016}, \cite{MGA}, \cite{BERNTSEN2017886}, \cite{Yavuz2011}, \cite{Optimum_not_enough}.

The MGA technique introduced by \cite{Brill_MGA_1982} and implemented on an energy-economic model by \cite{DeCarolis_MGA}, is referred to as the Hop Skip Jump (HSJ) MGA algorithm, will produce a small number of alternative solutions from the feasible near optimal decision space.
These alternative solutions do provide some insights in the characteristics of the feasible near optimal decision space, but a complete picture is not given. Furthermore, the solutions found when using the HSJ MGA algorithm are somewhat randomly located in the feasible near optimal decision space, and the found solutions are highly dependent on the starting point. 

In this project the MGA approach will be further explored in an attempt to map the entire volume of the feasible near optimal solution space, and hereby providing a detailed description of all possible outcomes of an energy-economic model. This will provide greater insights, as knowing the shape of the feasible near optimal space provides the opportunity to create histograms and probability density functions highlighting capacity ranges most likely to be feasible amongst other information.

Maybe something about how to map the feasible near optimal space 

% Model concept
In this project the model presented in: \cite{PyPSA_euro_30_model} of the European electricity grid, will serve as the base model. The model is build in \cite{Pypsa}, and formulates as a techno-economic linear optimization problem, with the objective of minimizing total annual system cost, while satisfying a range of constraints ensuring feasible operation. The model groups the European electricity network into 30 nodes, each one representing a single country. Countries are linked with power lines approximating the current layout of the European transmission grid. Each node in the network, will in this project, only be granted access to three electricity generating technologies and no storage technologies, simplifying the network drastically compared to the configuration used in \cite{PyPSA_euro_30_model}. The energy generating technologies chosen are open cycle gas turbines (OCGT), wind and solar power.

% Approach overview
The goal of this project is to develop a method capable of exploring the volume of the feasible near optimal decision space from such linear techno-economic model, in order to extract probability data regarding installed capacities, technology combinations etc. 

- Approach for developing method

- Very explicit explanation of how energy system optimization is performed now
- Explain what is new about this method 

\begin{comment}
- Increased demand and a need to reduce CO2 emmisions 
- Fundamental change is needed (policy wise)
- Energy-economy models is an important tool 
- Modelers should focus on robust insights rather than point estimates
- Uncertainty in the models (Structural and parametric)
- Structural uncertainty is addressed with higher comlexity
- Parametric uncertainty is addressed with running multiple scenarios or sensitivity analysis
- Scenario approach does not include less expected real-world developments 
- Little to none possibility to validate models 
\end{comment}





\begin{comment}
- MGA \cite{Brill_MGA_1982}
- Feasible near optimal decision space 

Alternative solutions generated with energy-economy
optimization models also provide valuable insight that can be used to
challenge preconceptions and suggest creative alternatives to decision makers.
\end{comment}

%\subsection{project boundary's}


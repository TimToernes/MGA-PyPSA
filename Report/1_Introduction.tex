
\chapter{Introduction}


%Starting your introduction
% Large Scale Background
As climate change and the effects thereof have become more and more evident in recent years, the effort to reduce the emission of greenhouse gases has increased. This comes to show in ambitious goals of reduced emissions on a national and international scale. The majority of greenhouse gas emissions arise from the energy sector including transportation, heating/cooling, and electricity demand \cite{eea_co2_emission}, therefore efforts must be put into reducing emissions from the energy sector. 

%Topic and context

To achieve reductions in greenhouse gas emissions from the energy sector, large amounts of variable renewable energy sources, such as wind and solar energy is being implemented in the energy grid. The volatile nature of these variable renewable energy sources drives the need for collaboration between countries and energy sectors to handle periods of scarce resources and fluctuations in energy demand. 
This complicates the already complex task of energy system synthesis even further, hereby requiring decision-makers to have greater in-depth knowledge. Therefore, the need for analysis tools providing insights into the constraints and possibilities of the complex energy systems has never been more present.  

% Energy economic models 
A frequently used tool to gain insight in future energy grid compositions is numeric techno-economic energy models on either regional, national or international scale. These models can be used to study the behavior and composition of existing and future energy networks, together with the impact of new technologies or structural changes in the networks \cite{Gorm_impact_of_CO2_PYPSA}. The models often approximate the behavior of the physical systems, through a series of mathematical constraints based on energy laws. By inputting time series for wind and solar availability and energy demand, the behavior of the energy network can be studied with alternative configurations of energy generating and storage technologies. Using an optimization algorithm, it is possible to find the cheapest configuration of the energy network that satisfies all user prescribed constraints. 

% Model uncertainties 
Techno-economic energy models do, however, suffer from large uncertainties and the lack of validation possibilities, resulting in unreliable and therefore less informative results.
Model uncertainty can either arise from uncertainty in input parameters and data, such as uncertainties in future technology prices and energy demand. This type of uncertainty is referred to as parametric uncertainty. Parametric uncertainty is well understood and often studied with sensitivity analysis or Monte Carlo simulations. 
A different type of uncertainty introduced by an incomplete or faulty mathematical description of the problem has been found to have just as large effects on the results produced by these techno-economic models \cite{DeCarolis_MGA}. As this type of uncertainty relates to the mathematical foundation or structure of the model, they are referred to as structural uncertainty. 
Structural uncertainty is inevitable as it is impossible to create completely accurate mathematical models representing physical systems. A common source of structural uncertainty is unmodeled objectives such as public acceptance issues or political ambitions. Furthermore, the economy of energy systems, which is often the parameter to be optimized, is highly influenced by politics and public opinion. It is easy to imagine that a scenario with large amounts of aerial electricity transmission lines would be met with public resistance, and therefore a scenario with less transmission at the same cost might be favorable. 
This does in term mean that the model objective is not only to reduce system cost but also to satisfy as many of the involved stakeholders as possible. It is however impossible to model the satisfaction of all involved stakeholders, let alone the challenge of identifying all future parties involved.  

%Focus and scope

% Literature review/Current standpoint
Recently an approach addressing the structural uncertainty of the techno-economic models has been proposed by J. DeCarolis. In an article \cite{DeCarolis_MGA} a technique called Modeling to Generate Alternatives (MGA), from the field of management research/planning science \cite{Brill_MGA_1982}, is applied to the field of energy planning. The root cause of structural uncertainty cannot be addressed as it can with parametric uncertainty, as the origin of structural uncertainty is hard to define. Instead, one must investigate all solutions near the one found to be optimal, and estimate the likelihood of these near-optimal solutions being the true optimal solution. In the technique proposed by J. DeCarolis, a finite set of maximally different near-optimal solutions are found. The difference in the found solutions can then be used as a measure of structural uncertainty and provides a variety of alternatives to the optimal configuration of the energy system. The concept of using MGA algorithms on energy planning problems have been further studied and the result presented in a range of articles and papers; \cite{DeCarolis_MGA}, \cite{DECAROLIS2016}, \cite{MGA_Price}, \cite{BERNTSEN2017886}, \cite{Optimum_not_enough} and \cite{Fabian_MGA}.

The MGA technique introduced by Brill et al. \cite{Brill_MGA_1982} and implemented on a techno-economic energy model by J. DeCarolis \cite{DeCarolis_MGA}, is referred to as the Hop Skip Jump (HSJ) MGA algorithm. It will produce a small number of alternative near-optimal solutions, by altering the objective function of the optimization algorithm. Instead of minimizing cost, the objective is changed such that the model seeks to implement previously unused technologies. In order to ensure that the found alternative solutions are near the optimum, a new constraint is added to the model. This constraint allows for the new MGA solutions to deviate in cost by a certain amount from the optimal solution. This constraint is referred to as the MGA constraint. 

The alternative solutions found with the HSJ MGA algorithm, do provide some insights into the characteristics of all near-optimal solutions, but it is not possible to determine if all possible solutions have been found. 
Furthermore, the HSJ method of finding alternative solutions is somewhat random and the found alternative solutions are highly dependent on the starting point. In order to fully understand the characteristics of all near-optimal solutions, a more structured method is needed.  


% Model concept
In this project the model presented in \cite{PyPSA_euro_30_model} of the European electricity grid, will serve as a base model, to use for testing and validation. The model is built in the open-source framework PyPSA \cite{Pypsa}, and formulated as a techno-economic linear optimization problem. The objective in the model is to minimize total annual system cost while satisfying a range of constraints ensuring feasible operation. The model groups the European electricity network into 30 nodes, each one representing a single country. Countries are linked with power lines approximating the current layout of the European transmission grid. Each node in the network, will in this project, only be granted access to three electricity-generating technologies and no storage technologies, simplifying the network drastically compared to the configuration used in \cite{PyPSA_euro_30_model}. The energy-generating technologies chosen are open cycle gas turbines (OCGT), wind turbines and solar photovoltaic (solar PV).


%Relevance and importance

The current use of techno-economic models combined with optimization tools, not using MGA algorithms, provides a very rigid solution to the future configuration of energy networks. Usually, a single optimal solution is found and used as an end goal to strive towards. This method of doing energy planning provides little guidance, in the case of unforeseen events and changing objectives. Using the MGA approach presented by J. DeCarolis, it is possible to have a set of alternative near-optimal solutions providing guidance if the initial optimal solution, should become unachievable or unfeasible. Having the characteristics of all near-optimal solutions available, it would be possible to define a tolerance on the optimum, and to provide details on minimum required capacities and must-have technologies. Furthermore, it would be possible to provide a cost estimate for deviating from the optimal design, guiding decision-makers in the case of unforeseen events and changing objectives. Having this information would add value to techno-economic optimization studies, as the result would be relevant for longer periods of time, requiring less frequent reevaluations of the studies. 


%Questions and objectives
In this project the MGA concept will be further explored, in an attempt to map the entire set of near-optimal solutions, providing a detailed description of all feasible near-optimal configurations of a given techno-economic model. The objective of this project is to develop a structured method of investigating the characteristics of all feasible near-optimal solutions to a techno-economic optimization problem. The developed method should be verified on the techno-economic model from \cite{PyPSA_euro_30_model} and the quality of the insights provided should be discussed. 



%Overview of the structure
The structure of this project will be as follows. Initially, a detailed analysis of the mathematical formulation of a techno-economic model will be presented. The properties of the mathematical constraints used in the model are analyzed together with the objective function. Having established the formulation of the model, the characteristics of the set containing all near-optimal solutions will be presented. The use of optimization algorithms used to find optimal solutions in this set will briefly be discussed. With a good understanding of the techno-economic optimization problem, the working principles of existing MGA algorithms will be introduced followed by a presentation of the MGA method developed in this project. 
In chapter 3, the techno-economic model used in this project will be presented, explaining the implemented technologies and accounting for input data. The results of the computational experiments performed in this project, using the reference model and the presented MGA algorithm, will be presented in chapter 4. Several experiments have been performed highlighting the flexibility and usability of the proposed MGA algorithm. The found results will be discussed in chapter 5, analyzing the characteristics of the techno-economic model of Europe used in this project, together with an evaluation of the MGA algorithm itself. Finally, the project will be concluded upon, highlighting relevant findings and achieved objectives. 





\chapter{Discussion}

%Research problem/Key findings 
% - MGA approach is feasible
% - Multiplicity
%
% - There is a lower limit to decreasing OCGT capacity
% - As CO2 constraint tightens flexibility increases 
% - Expected correlations between wind and transmission is found
% - Gini coefficient increases as CO2 is reduced

% Research question
This chapter will critically examine the results found in this project and the assumptions made to achieve these results. The objective of this project has been to explore all near-optimal solutions of the techno-economic model of Europe presented, to determine common characteristics of these solutions. A method capable of mapping all near-optimal solutions in a structured manner has been developed, and the usefulness of this method and its performance will be critically examined in this chapter. 

The method developed to search the near-optimal feasible space has been applied to the problem of techno-economic energy system optimization on an international scale. There is however nothing preventing this method from being used in other fields of numeric modeling and optimization, as the method itself is formulated in general terms. The method developed does, however, require that the investigated problem is linear and convex. 

% Quick recap
Initially, the formulation of the techno-economic model was studied, and based on key findings, a method capable of searching the near-optimal feasible space was proposed. Several studies have been performed with the proposed MGA method analyzing the presented techno-economic model of Europe. Key results of the studies show that the proposed method is capable of providing useful insights, revealing must-have technology capacities and simultaneously indicating where solution flexibility is available.

% Decision space complexity % Relevance of MGA
Analyzing the results achieved in both studies using the presented MGA algorithm in Section \ref{sec:4D} and \ref{sec:7D}, a vast complexity amongst the near-optimal solutions is found. The complexity of the near-optimal feasible spaces found using relatively small MGA slacks shows the importance of addressing structural uncertainty with MGA algorithms, as small flaws in the model formulation can result in solutions representing very different configurations of the energy network. 
Accepting an uncertainty in total system cost of a very large public planning project, such as modernizing the entire European electricity grid, to be at least 10\% is not unrealistic. This in term means that all solutions found in the MGA studies using a 10\% slack on total system cost are equally realistic. Analyzing the presented results from both studies using the MGA method on the full Europe model in Section \ref{sec:4D} and \ref{sec:7D}, very large spans in technology capacities are found. In Figure \ref{fig:4d_hist}, it is seen that in a scenario where Europe decreases $\text{CO}_2$ reductions with 95\%, an energy network with 1000GW of installed wind power across Europe is just as feasible as a scenario with 2000GW of installed wind power. These are huge spans, and that is why a single optimal solution is far from enough when modeling future scenarios of energy systems. 

% MGA compared to previously used methods
The results achieved through the use of MGA provides the decision-maker with a framework exhibiting the possibilities available with a given system. This allows for an agile process when synthesizing future energy systems as a continuous range of possible energy systems are made available. Instead of striving towards a single optimal solution the decision-maker is capable of assessing the desires of all stakeholders and designing an energy solution within the frames presented in the MGA results, satisfying as many individual objectives as possible. 
Results generated with previously presented MGA methods such as the work presented by J. DeCarolis in \cite{DeCarolis_MGA}, where a small set of alternative solutions to a techno-economic optimization problem is found by using the HSJ MGA algorithm, can be rather complex to understand. Grasping over a large number of alternative solutions at once is a very complex task, and by generating only a small number of alternative solution the algorithm fails to present all the possibilities within the near-optimal feasible space. With the MGA algorithm presented in this project a continuous span of alternative solutions is found, making it simpler for the decision-maker to alter the desired solution such that it satisfies as many stakeholders as possible. 
A critical flaw in previously presented MGA algorithms such as the HSJ algorithm used in \cite{DeCarolis_MGA}, and the maximize/minimization of all variables used in \cite{Fabian_MGA}, is that there is no guarantee that all possible alternative solutions have been exposed. The nature of the proposed MGA algorithm from this project ensures that it converges towards the full solution as it iterates. This provides a guarantee that all possible configurations of the energy network are well investigated, removing any bias that the other MGA methods might have. 

%\subsection{Performance of MGA algorithm }
The benefits of using MGA are clearly shown in this report, but using increasingly complex algorithms does however come at a cost. The MGA algorithm presented in this project introduces a significant increase in the computational time needed for a single study. In the MGA study searching a decision space in 7 dimensions more than 2000 optimizations of the techno-economic problem were performed. This requires much more computing time than classic optimization where a single or perhaps a few optimizations are performed. Previously presented MGA algorithms such as the HSJ \cite{DeCarolis_MGA} algorithm rely on a handful of optimizations usually less than 10, to generate its results. Therefore, one must consider if the gained insights are worth the added computational time. But with cloud computing becoming more and more widespread and prices for computing time as low as 0.64\$ per hour for a 64 core compute note with 240GB of memory \cite{Google_Cloud} on Google's cloud computing platform, a very large increase in computational time can be accepted for gained insights. 

% Why high dim are so hard to deal with.
When increasing the number of variables included in the reduced decision space the computation time needed to perform an MGA study increases rapidly. There are mainly two reasons why the computational time needed increases. First of all, using the convex hull face normal's as searching directions, introduce a lot more search directions as the dimension of the convex hull increases. A higher-dimensional geometry simply has a more complex shape, and thereby more faces than a lower-dimensional one, thus more optimizations are needed. A different factor is that it is simply harder to compute the convex hull of a higher dimensional set of points. The computation time for the algorithm used to compute the convex hull of a finite set of points in this project \cite{qhull} increases at roughly $O^*(n^3)$. The result of this is that it is unfeasible to use the presented MGA algorithm on a decision space of very large dimension. The largest number of dimensions used for a successful MGA study in this project is seven, but it is estimated that a study using ten dimensions is feasible. 

% Configuration of reduced decision space
Due to the increasing computing time needed as the number of decision variables increases, the developed MGA algorithm is not capable of handling the entire decision space. Therefore it is necessary to limit the decision space considered by the MGA algorithm to include only a few variables. Selecting variables to include in the reduced decision space, allows the modeler to focus the MGA study on certain areas of interest. In this project, two studies with different focus have been performed providing information on two different subjects. In future MGA studies, one could imagine the variables included in the reduced decision space being technology capacities installed in a single country or one might focus on the capacities of individual transmission lines. When selecting variables to explorer, the modeler has the opportunity to focus the study on certain elements of the model, hereby introducing the opportunity to investigate the interplay between a single technology and the entire European electricity grid.  

% Outlook/Future work
When reducing the dimensions of the considered decision space by grouping variables together, the solutions located within this reduced decision space are no longer evenly distributed. In the reduced decision space, a single solution covers a range of possible solutions. This multiplicity of solutions in the reduced decision space was investigated in Section \ref{sec:Multiplicity}, where a method capable of estimating the effect of this feature on small problems was proposed and the implications of not considering multiplicity in larger problems were discussed. The effect of multiplicity is significant and to improve the proposed MGA method, future works should investigate methods capable of determining the multiplicity of larger problems. 

% Recomendations 


% - Small correlations mean large freedom 

% Must haves and real choices \cite{Optimum_not_enough}



% - Capable of showing feasibility of technologies 
% - Information about flexibility
% - Maybe multiplicity is not that imortant as it usualy just reinforces the results already found 


% - Variable demand / wind profiles 


% - Polytope does not nececarily apply to decision spaces of reduced dimension 
% - 
% - Is the use of a Convex hull as direction maker a good idea?
% - Discuss sampling method

%Give your interpretations
% - Small correlations mean large freedom 
% - 


%Discuss the implications
% - New model provides greater insight in flexibility and need to have capacities 
%
% - Larger unequality must be accepted or higher price paid as CO2 is decreased
%

%Acknowledge the limitations
% - Method can only handle ~7D and not capable of estimating multiplicity
%     - This might not be as severe as maximum and minimum ranges are stil found 
% 
% - Europe model is coars and to few technologies included 


% State your recommendations
% - Use this method on model with storage and more technologies 
% - 
% 




% - Why high dim are so hard to deal with.

% - Adressing structural and parametric uncertainty